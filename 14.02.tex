\section{Продолжение истории с матрицами}
Наша текущая мотивация: компактно и точно описывать системы линейных уравнений(далее СЛУ) на языке матриц.

Для этого заметим, что в СЛУ каждая строчка имеет следующий вид: $a_1x_1 +\dots+ a_nx_n$.
Поэтому давайте заведём следующую операцию умножения для матриц $M_{1\times n}(R) \times M_{n\times 1}(R)\rightarrow R$(в данном контексте и далее $R$~---~произвольное кольцо),
а именно: 
$$
\begin{pmatrix}
    a_1&\dots&a_n
\end{pmatrix} \times
\begin{pmatrix}
    x_1\\\vdots\\x_n
\end{pmatrix}=
a_1x_1 +\dots+ a_nx_n$$

\begin{defn}[Вектор]
    \textbf{Вектором} размера $n$ будем называть матрицу $M_{n\times 1}$.
\end{defn}

Теперь логично захотеть определить умножение матрицы на вектор $M_{m\times n}\times M_{n\times 1}\Rightarrow M_{m\times 1}$ таким образом, чтобы удобно было получать нечто похожее на СЛУ.
Очевидно это можно сделать умножением каждой строчки на вектор следующим образом:
$$
\begin{pmatrix}
a_{1,1}&\dots&a_{1,n}\\
\vdots&\ddots&\vdots\\
a_{m,1}&\dots&a_{m,n}
\end{pmatrix}\times
\begin{pmatrix}
x_1\\
\vdots\\
x_n
\end{pmatrix} = 
\begin{pmatrix}
a_{1,1}x_1 +\dots+ a_{1,n}x_n\\
\vdots\\
a_{m,1}x_1 +\dots+ a_{m,n}x_n\\
\end{pmatrix}
$$
После этого уже можно обобщить наше произведение на две матрицы. Для этого представим, что 
мы умножаем систему уравнений не на набор неизвестных, получая матрицу, похожую на систему
уравнений, а на несколько наборов значений, ведь матрицу можно представить как
упорядоченный набор векторов.
\begin{defn}[Произведение матриц]
    Произведение матриц $A\in M_{m\times n}, B\in M_{n\times k}$ определим следующим образом:
    $$
    A\times B = 
    A\times \big( B_{1,*},\dots,B_{k,*} \big) = 
    \big(A\times B_{1,*}, \dots, A\times B_{k,*}\big)=
    \begin{pmatrix}
        a_{1,1}b_{1,1} +\dots+ a_{1,n}b_{1,n} & \dots & a_{1,1}b_{k,1} +\dots+ a_{1,n}b_{k,n}\\
        \vdots & \ddots & \vdots\\
        a_{m,1}b_{1,1} +\dots+ a_{m,n}b_{1,n} & \dots & a_{m,1}b_{k,1} +\dots+ a_{m,n}b_{k,n}\\
    \end{pmatrix}
    $$
    Иначе говоря, если $A\times B = C$, тогда: $$C_{i,j} = \sum_{k=0}^{n}A_{i,k}B_{k,j}$$
\end{defn}
\rem
    (P.s знак $\times$ при умножении матриц и векторов в будущем часто будет опускаться)
\rem
    Произведение матриц не коммутативно: $$
    \begin{pmatrix}
        1 & 2\\
        3 & 4
    \end{pmatrix}
    \begin{pmatrix}
        0 & 1\\
        1 & 0
    \end{pmatrix}=
    \begin{pmatrix}
        2 & 1\\
        4 & 3
    \end{pmatrix};\quad
    \begin{pmatrix}
        0 & 1\\
        1 & 0
    \end{pmatrix}
    \begin{pmatrix}
        1 & 2\\
        3 & 4
    \end{pmatrix}=
    \begin{pmatrix}
        3 & 4\\
        1 & 2
    \end{pmatrix} $$
\begin{examples}
    Давайте приведём несколько примеров использования введённой нами операции.\\
    \begin{enumerate}
        \item Для записи СЛУ в виде $Ax = b$, где $A$~---~матрица, а $x,b$~---~вектора.
        \item Для записи скалярного произведения: $a^Tx$, где $x, a$~---~ вектора, а
            $a^T$ операция транспонирования матрицы, о которой будет сказано позже. Пока
            её можно воспринимать как переворот вектора в матрицу $M_{1\times n}$.
        \item
            Для подсчёта количества путей размера $k$ между любыми двумя вершинами 
            ориентированного графа.

            Пусть $G$~---~ориентированный граф. Тогда его \textbf{матрицей смежности} $A(G)$ определим следующим образом:
            $$ A(G)_{i,j}
            \begin{cases}
                1, \text{если есть ребро $i\rightarrow j$}\\
                0, \text{иначе}\\
            \end{cases}
            $$
            Тогда легко показать, что $(A(G)_{i,j})^k$~---~количество путей $i\rightarrow j$ длины $k$.
        \item
            Для подсчёта суммы путей размера $k$ между любыми двумя вершинами взвешенного
            ориентированного графа, где длинна пути считается как произведение весов рёбер на нём.

            Пример такой задачи: есть статистика по переселению людей между $n$ городами в следующем виде:
            известно, что вероятность переезда человека из города $i$ в город $j$ равна весу ребра $i\rightarrow j$.
            Надо по этим данным научиться узнавать шанс того, что человек из города $i$ через $k$ лет будет жить
            в городе $j$.

            Для простоты будем считать, что сумма весов исходящих рёбер для каждой вершины равна $1$. Если это
            не так, то мы можем это исправить, создав ребро $i\rightarrow i$ c нужным нам весом.
            В данной формулировке наиболее понятно, что шанс переезда человека, живущего в городе $i$, в город $j$
            через $k$ лет будет равен сумме весов путей $i\rightarrow j$ длины $k$, если вес пути это произведение
            всех рёбер на нём. Таким образом, мы получили задачу похожую на предыдущую, за исключением того, что
            наш граф теперь взвешенный, а значит $(A(G)_{i,j})^k$~---~искомая вероятность.
    \end{enumerate}
\end{examples}
\begin{properties}\leavevmode
    \begin{enumerate}
        \item
            Ассоциативность произведения матриц в случае, где оба произведения определены:
            $(AB)C = A(BC)$. (Доказать самим)
        \item
            Существование единичной матрицы $E_n = 
            \begin{pmatrix}
                1 & 0 & 0 & \dots & 0\\
                0 & 1 & 0 & \dots & 0\\
                \vdots & \vdots & \vdots & \ddots & \vdots\\
                0 & 0 & \dots & 0 & 1
            \end{pmatrix}$
            Доказательство того, что она не изменяет умножаемую на неё, очевидно.
        \item Обратный элемент существует не для любой матрицы. Приведём пример с 
            $
            \begin{pmatrix}
                0 & 0\\
                0 & 1
            \end{pmatrix}$: $$
            \begin{pmatrix}
                0 & 0\\
                0 & 1
            \end{pmatrix}
            \begin{pmatrix}
                a & b\\
                c & d
            \end{pmatrix} = 
            \begin{pmatrix}
                0 & 0\\
                c & d
            \end{pmatrix}\not=
            \begin{pmatrix}
                1 & 0\\
                0 & 1
            \end{pmatrix} $$
    \end{enumerate}
\end{properties}
\defn[Сложение матриц] Введём очевидную операцию сложения матриц: $(A+B)_{i,j} = A_{i,j} + B_{i,j}$.
\begin{properties}\leavevmode
    \def\zero{\textbf{0}}
    \begin{enumerate}
        \item
            $A + B = B + A$
        \item
            $(A + B) + C = A + (B+C)$
        \item
            $\zero_{m\times n} =
            \begin{pmatrix}
                0 & \dots & 0\\
                0 & \ddots & 0\\
                0 & \dots & 0
            \end{pmatrix}$
        \item
            $\forall A \in M \exists (-A): A + (-A) = \zero$
        \item
            $(A+B)C = AC + BC$\\
            $C(A+B) = CA + CB$
    \end{enumerate}
\end{properties}

\defn Введём операцию умножения вектора на скаляр следующим образом:
$$ \lambda
\begin{pmatrix}
    x_1\\
    \vdots\\
    x_n
\end{pmatrix} =
\begin{pmatrix}
    \lambda x_1\\
    \vdots\\
    \lambda x_n
\end{pmatrix} $$
\begin{properties}\leavevmode
    \newcommand\xvec{
        \begin{pmatrix}
            x_1\\
            \vdots\\
            x_n
        \end{pmatrix}
    }
    \begin{enumerate}
        \item
            $$
            \begin{pmatrix}
                \lambda & 0 & 0 & \dots\\
                0 & \lambda & 0 & \dots\\
                \vdots & \vdots & \ddots & \vdots\\
                0 & \dots & 0 & \lambda
            \end{pmatrix}
            \xvec
            =
            \begin{pmatrix}
                \lambda x_1\\
                \vdots\\
                \lambda x_n
            \end{pmatrix} = 
            \lambda
            \xvec
            $$
        \item
            $$
            (\lambda\mu)\xvec = (\mu\lambda)\xvec = \lambda\left(\mu\xvec\right)
            $$
    \end{enumerate}
\end{properties}
\defn Введём операцию умножения матрицы на скаляр следующим образом:
$$ \lambda A = 
\begin{pmatrix}
    \lambda & 0 & 0 & \dots\\
    0 & \lambda & 0 & \dots\\
    \vdots & \vdots & \ddots & \vdots\\
    0 & \dots & 0 & \lambda
\end{pmatrix} A $$
\begin{properties}\leavevmode
    \begin{enumerate}
        \item
            $$
            \lambda(AB) = (\lambda A)B = A(\lambda B)
            $$
    \end{enumerate}
\end{properties}
\begin{rem}
    Исходя из введённых выше операций и их свойств становятся очевидными некоторые утверждения
    про решения СЛУ, перечислим их:
    \begin{enumerate}
        \item
            $$
            \begin{cases}
                Ax = b\\
                Ay = c
            \end{cases}\Rightarrow
            A(x+y) = b+c
            $$
            Где $x,y,b,c$~---~вектора, а $A$~---~матрица системы.
        \item
            Из предыдущего свойства:
            $$
            \begin{cases}
                Ax=b\\
                Ax_0=0
            \end{cases}\Rightarrow
            A(x+x_0)=b\Rightarrow A(x + kx_0) = b, \forall k\in \N
            $$
        \item
            Пусть $L = \{x\mid Ax=0\}\subset R^n$, тогда знаем следующие факты про 
            $x_1, x_2\in L, \lambda \in R$:
            \begin{enumerate}
                \item $x_1 + x_2 \in L$
                \item $\lambda x_1 \in L$
            \end{enumerate}
    \end{enumerate}
\end{rem}
\section{Векторные пространства}
\subsection{Базовые понятия}
Заметим, что полученные нами свойства не вписываются в изучаемые ранее структуры, из-за чего возникает желание завести новый объект. 
\defn[Векторное пространство] \textbf{Векторным пространством} над полем $K$ будем называть набор $(V, +, \cdot)$, удовлетворяющий свойствам ниже.
\begin{properties}\leavevmode
    \begin{enumerate}
        \item 
            \begin{enumerate}
                \item Определена операция умножения вектора на скаляр $\cdot: K \times V \mapsto V$
                \item Определена операция сложения векторов $+: V \times V \mapsto V$
            \end{enumerate}
        \item $(V,+)$~---~абелева группа
        \item $\lambda(v_1 + v_2) = \lambda v_1 + \lambda v_2$
        \item $(\lambda + \mu)\cdot v = \lambda v_1 + \lambda v_2$
        \item $1\cdot v = v$
        \item $(\lambda\mu)v = \lambda(\mu v)$
    \end{enumerate}
\end{properties}
\begin{examples}\leavevmode
    \begin{enumerate}
        \item $K^n$~---~векторное пространство над $K$.
        \item $M_{m\times n}(K)$~---~векторное пространство над $K$.
        \item
            \begin{enumerate}
                \item $\C$~---~векторное пространство над $\R$.
                \item $\R$~---~векторное пространство над $\Q$.
                \item Таким образом можно показать, что если $K$ является подполем $L$, то $L$~---~векторное пространство над $K$.
            \end{enumerate}
        \item $C[a,b]$~---~векторное пространство над $\R$.
        \item $C'[a,b]$~---~векторное пространство(далее ВП) над $\R$.
    \end{enumerate}
\end{examples}
\defn[Подпространство] Если $W\subseteq V$, то $W$ является \textbf{подпространством} при выполнении следующих условий:
\begin{enumerate}
    \item $a+b\in W \quad \forall a, b\in W$.
    \item $\lambda\cdot v \in W \quad \forall \lambda \in K, v\in W$.
    \item $\overline{0}\in W$
\end{enumerate}
\example
$\{x\mid Ax = 0\}$ подпространство в $K^n$.
\defn[Линейная комбинация векторов] \textbf{Линейной комбинацией векторов} $v_1,\dots,v_n\in V$ назовём: $\lambda_1v_1 + \lambda_2v_2 + \dots + \lambda_nv_n$, где $\lambda_i\in K$.
\defn[Линейная зависимость] \textbf{Линейно зависимыми} назовём множество векторов такое, что любой вектор из множества
записывается в виде линейной комбинации остальных. Иначе говоря, $v_1,\dots,v_n$~---~линейно зависимы если:
$\exists \lambda_1,\dots,\lambda_n\in K: \lambda_1v_1 + \dots + \lambda_nv_n=0, \exists \lambda_j \not= 0$.
